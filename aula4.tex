%%%%%%%%%%%%
%% Aula 4 %%
%%%%%%%%%%%%

\part{Aula 4}
\section{Aula 4}

\subsection{Apresentações (BEAMER)}

\begin{frame}[fragile]{Introdução}
    \begin{itemize}
    \item {\footnotesize BEAMER} é uma classe do \LaTeX\ para criação de apresentações de slides ou transparências.
    \item Uma apresentação em {\footnotesize BEAMER} é criada como qualquer outro documento \LaTeX, diferentes slides são colocados em ambientes (chamados \textit{frames}).
    \item Muitos pacotes de \LaTeX\ já contêm a classe {\footnotesize BEAMER}, no entanto, sua versão atualizada pode ser encontrada em: \url{http://bitbucket.org/rivanvx/beamer}
    \end{itemize}
\end{frame}

\begin{frame}[fragile]{Vantagens}
    \begin{itemize}
    \item Pode ser usado com \verb|pdflatex|, \verb|latex+dvips|, \verb|luatex| e \verb|xelatex|.
    \item Efeitos e sobreposições podem ser criados facilmente.
    \item Sua estrutura torna fácil criar apresentações de outras classes como \verb|article| e \verb|book|.
    \item A saída final é um arquivo PDF, assim não é preciso se preocupar se determinado programa está instalado em diferentes locais.
    \end{itemize}
\end{frame}

\begin{frame}[fragile]{Estrutura básica}
    Uma apresentação em {\footnotesize BEAMER} tem a mesma estrutura de um documento \LaTeX.
    
{\footnotesize
\begin{verbatim}
    \documentclass{beamer}

    \usetheme{Darmstadt}

    \title{Titulo da apresentação}
    \author{Nome do autor}
    \institut{Instituição}
    \date{Data da apresentação}

    \begin{document}
        \frame{\pagetitle}

        \begin{frame}
            Minha apresentação
        \end{frame}
    \end{document}
\end{verbatim}
}
\end{frame}

\begin{frame}[fragile]{Frames}
    \begin{itemize}
    \item Frame é o ambiente onde se cria um ou uma sequência de slides.
    \item[] \alert{\textbackslash begin\{\textit{frame}\}[\textit{opções}]\{\textit{Título}\}}
    \item Eles também podem ser criados com o comando \alert{\textbackslash frame\{\}}.
    \item Se o conteúdo exceder um slide, a opção \alert{allowframebreaks} pode ser utilizada para dividir o frame em várias partes.
    \item Se a divisão não ficar como desejado, também pode-se usar o comando \alert{\textbackslash newpage} para começar um novo slide a partir deste ponto.
    \end{itemize}
\end{frame}

\begin{frame}[fragile]{Página de título}
    \begin{itemize}
    \item Uma página de título é composta por 4 atributos: \alert{\textbackslash title}, \alert{\textbackslash author}, \alert{\textbackslash institute} e \alert{\textbackslash date}.
    \item Para inserí-la usa se o comando \alert{\textbackslash titlepage}.
    \end{itemize}

    \begin{example}
    \begin{semiverbatim}
\\title\{Título da apresentação\}
\\author\{Nome do(s) autor(es)\}
\\institute\{Universidade/Empresa/...\}
\\date\{Opcional - o padrão é a data atual\}

\\frame\{
    \\titlepage
    \}
    \end{semiverbatim}
    \end{example}
\end{frame}

\begin{frame}[fragile]{Sumário}
    \begin{itemize}
    \item O comando para se criar um sumário é o mesmo que nos outros documentos, porém ele deve estar dentro de um frame
    \end{itemize}

    \begin{example}
    \begin{semiverbatim}
\\frame\{
    \alert{\\tableofcontents}
    \}
    \end{semiverbatim}
    \end{example}
\end{frame}

\begin{frame}[fragile]{Sobreposições e efeitos (\textit{Overlay})}
    \begin{itemize}
    \item Para se fazer efeitos com o texto, como aparecer após 1 clique, mudar de cor e sumir, pode-se utilizar
    \end{itemize}

    \begin{columns}
        
        \column{.3\textwidth}
        \begin{itemize}
        \item[]<1-| alert@1> \verb|\pause|
        \item[]<2-| alert@2> \verb|\only|
        \item[]<3-| alert@3> \verb|\visible|
        \item[]<4-| alert@4> \verb|\invisible|
        \item[]<5-| alert@5> \verb|\alt|
        \end{itemize}
        
        \column{.7\textwidth}
           
        \only<1>{
            \begin{semiverbatim}
\\begin\{itemize\}

\ \ \\item A

\ \ \\pause
    
\ \ \\item B

\\end\{itemize\}
            \end{semiverbatim}
        }

        \only<2>{
            \begin{semiverbatim}
Contador: \\only<1>\{1\} \\only<2>\{2\} \\only<3>\{3\} \\only<4>\{4\} \\only<5>\{5\}
            \end{semiverbatim}
        }

        \only<3>{
            \begin{semiverbatim}
\\visible<3>\{Este texto será visível somente no slide 3.\}
            \end{semiverbatim}
        }

        \only<4>{
            \begin{semiverbatim}
\\invisible<-2>\{Este texto fica invisível até o slide 2 e visível no restante.\} 
            \end{semiverbatim}
        }

        \only<5>{
            \begin{semiverbatim}
\\alt<5>\{Texto para o slide 2.\}\{Texto para o restante.\}
            \end{semiverbatim}
        }
    \end{columns}
\end{frame}

\begin{frame}[fragile]{Especificação de \textit{Overlay}}
    \begin{itemize}
    \item Tanto nos comandos descritos anteriormente quanto em outros comandos de personalização (ex. \verb|\textit|, \verb|\textbf|, \verb|\color|, \verb|\alert| e \verb|\item|) e nos ambientes é possível adicionar uma especificação de um \textit{overlay}.
    \end{itemize}

    \begin{description}
    \item[\textbackslash comando<n>]  ocorre somente no slide \textbf{n};
    \item[\textbackslash comando<-n>]  ocorre até no slide \textbf{n};
    \item[\textbackslash comando<n->]  ocorre do slide \textbf{n} até o final do frame;\
    \item[\textbackslash comando<n-m>]  ocorre do slide \textbf{n} até o slide \textbf{m} 
    \end{description}
\end{frame}

\begin{frame}[allowframebreaks,fragile]{Ambiente Block}
    \begin{itemize}
    \item Block é um ambiente que forma uma caixa colorida ao redor do conteúdo, útil para destacar informações importantes.
    \item Existem vários ambientes Block diferentes para, principalmente, apresentações científicas: \verb|block|, \verb|example|, \verb|proof|, \verb|theorem|, \verb|alertblock|, \verb|definition|.
    \end{itemize}

\newpage

    \begin{block}{Título do bloco}
\begin{semiverbatim}
\\begin\{block\}\{Título do bloco\}
  - conteúdo -
\\end\{block\}
\end{semiverbatim}
    \end{block}
   
   \begin{alertblock}{título do bloco}
\begin{semiverbatim}
\\begin\{alertblock\}\{Título do bloco\}
  - conteúdo -
\\end\{alertblock\}
\end{semiverbatim}
    \end{alertblock}

    \begin{example}
\begin{semiverbatim}
\\begin\{example\}
  - conteúdo -
\\end\{example\}
\end{semiverbatim}
    \end{example}

\end{frame}

\begin{frame}[fragile]{Figuras e tabelas}
    \begin{itemize}
    \item Para criar tabelas e figuras no {\footnotesize BEAMER} usa-se os mesmos comando que em documentos.
    \end{itemize}

    \begin{example}
\begin{semiverbatim}
\small \% Figuras
\\includegraphics<2->[width=.5\\textlinewidth]\{imagem.jpg\}

\% Tabelas
\\begin\{tabular\}\{|c|c|\}
    \\invisible<1>\{X\} \& 0 \\
    \\hline
    0 \& \\invisible<1>\{X\}
\\end\{tabular\}
\end{semiverbatim}
    \end{example}

\end{frame}

\begin{frame}{Colunas}
\end{frame}


\subsection{Fórmulas Matemáticas}
\begin{frame}{Fórmulas Matemáticas}

\end{frame}


