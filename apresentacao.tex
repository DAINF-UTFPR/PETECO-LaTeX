\documentclass{beamer}

\usetheme{Darmstadt}
%\usefonttheme[onlylarge]{structerebold}
\setbeamerfont*{frametitle}{size=\normalsize,series=\bfseries}
\setbeamertemplate{navigation symbols}{}

\usepackage[brazil]{babel}
\usepackage[utf8]{inputenc}
%\usepackage{times}
\usepackage[T1]{fontenc}

\title{Minicurso \LaTeX}
\author{PET-ECO}
\institute{Universidade Teconlógica Federal do Paraná}
\date{Curitiba, PR - Março 2011}

\pgfdeclareimage[height=0.5cm]{university-logo}{utfpr.jpg}
\logo{\pgfuseimage{university-logo}}


\AtBeginPart
{
    \frame{\partpage}
    
	\begin{frame}[allowframebreaks]{Sumário}
			\tableofcontents[part=\insertpartnumber]
	\end{frame}
}


\begin{document}

\begin{frame}
	\titlepage
\end{frame}

%%%%%%%%%%%%
%% Aula 1 %%
%%%%%%%%%%%%

\part{Aula 1}
\section{Aula 1}

%% O que é LaTeX %%
\subsection{O que é \LaTeX?}

\begin{frame}{O que é \LaTeX?}
    \begin{itemize}
    \item \LaTeX (ou "LaTeX") é um sistema de preparação de documentos com alta qualidade tipográfica.
    \item É comumente utilizado para médios a largos documentos técnicos ou científicos, mas pode ser utilizado para quase qualqer tipo de publicação.
    \item \LaTeX é pronunciado "Lay-tech" ou "Lah-tech"
    \end{itemize}
\end{frame}

%% Histórico %%
\subsection{Histórico}

\begin{frame}{Histórico}
    \begin{itemize}
    \item Em 1978 Donald E. Knuth começou a desenvolver uma linguagem cujo objetivo era permitir a qualquer um formatar textos com muitas equações e com alta qualidade de saída, chamada de \TeX.
    \item Em 1985 Leisle Lamport desenvolveu um conjunto de macros denominado \LaTeX, que simplifica o uso da linguagem \TeX.
    \item Agora este projeto é mantido e desenvolvido pelo \LaTeX3 Project.
    \end{itemize}
\end{frame}

%% Conceituação %%
\subsection{Conceituação}

\begin{frame}{Conceituação}
    \begin{itemize}
    \item<alert@1> O \LaTeX\ não é um processador de textos!
    \item O \LaTeX\ enconraja o autor a não se preocupar miuto com a aparência e se focar na preparação do mesmo.
    \item Entretanto, algumas ferramentas — como o LyX — combinam o \LaTeX\ com a sistemática do WYSIWYG (What You See Is What You Get).
    \end{itemize}
\end{frame}

%% Vantagens e Desvantagens %%
\subsection{Vantagens e Desvantagens}

\begin{frame}[allowframebreaks]{Vantagens e Desvantagens}
    \begin{columns}
    \column{12cm}
    \begin{block}{Vantagens}
        \begin{itemize}
        \item Aparência profissional;
        \item A atenção dos usuários se concentra no conteúdo e não na aparência;
        \item Possibilidade de fácil utilização, devido ao uso de ferramentas gráficas como Kyle e LyX;
        \item A edição de fórmulas matemáticas é robusta e sua apresentação, visualmente agradável;
        \item Facilidade na criação de estruturas complexas como bibliografia, notas de rodapé, sumário e citações estão abstraídas;
        \item Ambos \TeX\ e \LaTeX\ são programas livres;
        \item Utilização modesta dos recursos do sistema.
        \end{itemize}
    \end{block}
    \end{columns}

    \begin{block}{Desvantagens}
        \begin{itemize}
        %\item \LaTeX\ does not work well for people who have sold their souls ...(haha)
        \item A principal desvantagem é que a criação de novos modelos leva muito tempo;
        \item A aprendizagem é mais difícil que em programas WYSIWYG, pois embora a estrutura lógica do documento seja intuitiva, os comandos do LaTeX, obviamente, não o são.
        \end{itemize}
    \end{block}
\end{frame}

%% Instalação no Linux %%
\subsection{Instalação no Linux}

\begin{frame}{Instalação no Linux}
    \begin{itemize}
    \item Depende de cada distribuição, sendo o caminho mais fácil procurar pelos pacotes \textit{texlive} e \textit{latex-beamer} no gerenciador de pacotes da distribuição utilizada.
    \end{itemize}
\end{frame}

%% Instalação no Windows %%
\subsection{Instalação no Windows}

\begin{frame}{Instalação no Windows}
    Passos para a instalção do \LaTeX\ no Windows:
    
    \begin{enumerate}
    \item Baixar o MikTex (Ambiente \LaTeX\ para windows)
    \item Baixar o WinEdt (Um editor \LaTeX, o melhor para windows, porém é shareware). Ou o TeXnicCenter (que é gratuito) ou qualquer outro editor de texto.
    \item Instalá-los na mesma ordem.
    \end{enumerate}
\end{frame}

%%%%%%%%%%%%
%% Aula 2 %%
%%%%%%%%%%%%

\part{Aula 2}
\section{Aula 2}

%% Arquivo de Entrada %%
\subsection{Arquivo de entrada}

\begin{frame}{Arquivo de entrada}
	\begin{itemize}
	\item A entrada para o \LaTeX\ é um arquivo de texto, ASCII ou UTF-8. Sendo possível criá-lo em qualquer editor de texto.
	\end{itemize}
	
	\alert{\it nomedoarquivo.tex \\ nomedoarquivo.bib} (Aula 3)

	\vspace{.5cm}
	\pause

	\begin{itemize}
	\item Arquivos auxiliares:
	\end{itemize}
	
	{\it nomedoarquivo.\{aux, log, nav, out, snm, toc, \dots\} }
\end{frame}

%% Estrutura básica %%
\subsection{Estrutura Básica}

% Comandos %
\begin{frame}{Comandos}
	\begin{itemize}
	\item Os comandos no \LaTeX\ têm a seguinte forma:
	\end{itemize}

	\alert{\textbackslash nomedocomando\{\textit{parâmetros(opcional)}\} }
	
	\vspace{.5cm}

	\begin{example}
		\textbackslash LaTeX \\
		\textbackslash section\{Introdução\}		
	\end{example}
\end{frame}

% Caracteres Especiais %
\begin{frame}{Caracteres especiais}
	\begin{itemize}
	\item Os seguintes símbolos são caracteres especiais no \LaTeX:
	\end{itemize}

	\alert{\# \$ \% \^{} \& \_ \{ \} \~{} \textbackslash}

	\begin{itemize}
	\item Para utilizá-los é necessário a inserção de ``\textbackslash'' antes:
	\end{itemize}

	\alert{
	   \textbackslash\# 
	   \textbackslash\$ 
	   \textbackslash\% 
	   \textbackslash\^{} 
	   \textbackslash\& 
	   \textbackslash\_ 
	   \textbackslash\{ 
	   \textbackslash\} 
	   \textbackslash\~{} 
	   \textbackslash textbackslash
	}
\end{frame}

% Espaçamento e parágrafos %
\begin{frame}[fragile]{Espaçamento e parágrafos}
	\begin{itemize}
	\item Apenas um espaço em branco é considerado pelo \LaTeX, para adicionar mais espaços é necessário inserir o comando \alert{``\textbackslash\ ''}.
	\item O comando \alert{\textbackslash\textbackslash} faz uma quebra-de-linha.
	\item Uma linha em branco representa um novo parágrafo.
	\end{itemize}

    
    \begin{block}{Entrada}
        \small
        \begin{semiverbatim}
\small{}Não faz diferença um ou mais        espaços depois de uma
palavra. Com \\\\ o texto vai para a próxima linha.

E uma linha em branco representa um novo parágrafo.
\end{semiverbatim}
    \end{block}    
    
    \begin{block}{Saída}
        \small
        Não faz diferença um ou mais       espaços depois de uma palavra. Com \\ o texto vai para a próxima linha.

        E uma linha em branco representa um novo parágrafo.
    \end{block}
\end{frame}

%% Estrutura do arquivo %%
\subsection{Estrutura do arquivo}

% Classe %
\begin{frame}[allowframebreaks]{Classe}
	\begin{itemize}
	\item Primeiro comando do arquivo deve ser o tipo do documento, ou classe, que é feito pelo seguinte comando:
	\end{itemize}

	\alert{\textbackslash documentclass[\textit{opções}]\{\textit{classe}\}}

	\vspace{.5cm}
	
	\begin{block}{Mais usados}
		\begin{center}
		\begin{tabular}{|l|l|}
		\hline
		{\bf Opções}				& {\bf Classes} \\
		\hline
		10pt, 11pt, 12pt			& article		\\
		a4paper, letterpaper, ...	& report		\\
		onecolumn, twocolumn		& book			\\
		twoside, oneside			& beamer		\\
		\vdots						& \vdots		\\
		\hline
		\end{tabular}
		\end{center}
	\end{block}
\end{frame}

% Pacotes %
\begin{frame}{Pacotes}
	\begin{itemize}
	\item É possível adicionar pacotes para aumentar as funcionalidades do \LaTeX\, como cores, figuras, fontes, etc. Para isso	se usa o comando:
	\end{itemize}

	\alert{\textbackslash usepackage[\textit{opções}]\{\textit{pacote}\}}

	\vspace{.5cm}

	\begin{example}
		\begin{itemize}
		\item \textbackslash usepackage[\textit{brazil}]\{\textit{babel}\}
		\item \textbackslash usepackage[\textit{utf8}]\{\textit{inputenc}\}
		\item \textbackslash usepackage[\textit{T1}]\{\textit{fontenc}\}
		\item \textbackslash usepackage\{\textit{graphicx}\}
		\end{itemize}
	\end{example}
\end{frame}

% Corpo do texto %
\begin{frame}[fragile]{Corpo do texto}
    \begin{itemize}
    \item Após feitas as configurações, o corpo do texto é iniciado pelo comando
    \end{itemize}

    \alert{\textbackslash begin\{\it document\}}
    
    \begin{itemize}
    \item E finalizado por
    \end{itemize}

    \alert{\textbackslash end\{\it document\}}

    \begin{example}
        \begin{semiverbatim}
\\documentclass[11pt,a4paper]\{article\}
\\usepackage[brazil]\{babel\}
\\usepackage\{amsmath\}
...
\\begin\{document\}
...
\\end\{document\}
\end{semiverbatim}
    \end{example}
\end{frame}

%% Comandos Globais de Configuração %%
\subsection{Comandos Globais de configuração}

% Capítulos, seções, subseções %
\begin{frame}{Capítulos, seções, subseções}
    \begin{itemize}
    \item O \LaTeX\ suporta até três níveis de seção e dois níveis de parágrafo na classe ``article'':
    \end{itemize}
    
    \alert{\textbackslash section\{...\} \\
           \textbackslash subsection\{...\} \\
           \textbackslash subsubsection\{...\} \\
           \textbackslash paragraph\{...\} \\
           \textbackslash subparagraph\{...\} \\
        }
    
    \begin{itemize}
    \item Em adicional, nas classes ``report'' e ``book'' há um seção superior
    \end{itemize}

    \alert{\textbackslash chapter\{...\}}
\end{frame}

% Título e Sumário %
\begin{frame}{Título e sumário}
    \begin{itemize}
    \item Para gerar o título de todo o documento usa-se
    \end{itemize}

   \alert{\textbackslash maketitle}

    \begin{itemize}
    \item E seu conteúdo é definido pelo seguintes comandos
    \end{itemize}

    \alert{\textbackslash title\{...\}} \\
    \alert{\textbackslash author\{...\}} \\
    \alert{\textbackslash date\{...\}} {\small (opcional)}

    \begin{itemize}
    \item Para construir o sumário, lista de figuras ou tabelas é necessário apenas um comando
    \end{itemize}
    
    \alert{\textbackslash tableofcontents \\
           \textbackslash listoffigures \\
           \textbackslash listoftables
        }

\end{frame}

% Ambientes %
\begin{frame}{Ambientes}
    \begin{itemize}
    \item Para compor textos com algúm propósito especial o \LaTeX\ define muitos tipos de ambientes
    \end{itemize}

    \alert{\textbackslash begin\{\it ambiente\} \\
           \textit{texto} \\
           \textbackslash end\{\it ambiente\}}

    \begin{itemize}
    \item É possível colocar vários ambientes um dentro do outro
    \end{itemize}

    \alert{\small
    \textbackslash begin\{aaa\} \\
    \ \ \textbackslash begin\{bbb\} \\
    \ \ \ \ \textbackslash begin\{ccc\} \\
    \ \ \ \ \vdots \\
    \ \ \ \ \textbackslash end\{ccc\} \\
    \ \ \textbackslash end\{bbb\} \\
    \textbackslash end\{aaa\}
    }
\end{frame}

% Listagens %
\begin{frame}[fragile]{Listagens}
    \begin{itemize}
    \item Existem três ambientes básicos para listagens:
        \begin{description}
        \item[itemize:] listas simples
        \item[enumerate:] listas enumeradas
        \item[description:] descrições
        \end{description}
    \end{itemize}
    
    \begin{columns}
        \column{.5\textwidth}
        \begin{block}{Entrada}
            \begin{semiverbatim}
\footnotesize{}\\begin\{enumerate\}
\\item Primeiro
  \\begin\{itemize\}
  \\item Com ponto
  \\item[-] Com traço
  \\end\{itemize\}
\\item Segundo
  \\begin\{description\}
  \\item[Item] descrição
  \\end\{description\}
\\end\{enumerate\}
\end{semiverbatim}
        \end{block}
    
        \column{.5\textwidth}
        \begin{block}{Saída}
            \begin{enumerate}
            \item[1] Primeiro
            \begin{itemize}
            \item Com ponto
            \item[-] Com traço
            \end{itemize}
            \item[2] Segundo
            \begin{description}
            \item[Item] descrição
            \end{description}            
            \end{enumerate}
        \end{block}
    \end{columns}
\end{frame}

% Alinhamento %
\begin{frame}[fragile]{Alinhamento}
    \begin{itemize}
    \item Existem três ambientes básicos para alinhamento:
        \begin{description}
        \item[flushleft:] alinha o texto a esquerda (default)
        \item[flushright:] alinha o texto a direita
        \item[center:] centraliza o texto
        \end{description}
    \end{itemize}
    
    \begin{columns}
        \column{.5\textwidth}
        \begin{block}{Entrada}
            \begin{semiverbatim}
\footnotesize{}\\begin\{flushleft\}
Texto alinhado à esquerda.
\\end\{flushleft\}

\\begin\{flushright\}
Texto alinhado à direita.
\\end\{flushright\}

\\begin\{center\}
Texto centralizado.
\\end\{center\}
\end{semiverbatim}
        \end{block}
    
        \column{.5\textwidth}
        \begin{block}{Saída}
            \begin{flushleft}
            Texto alinhado à esquerda
            \end{flushleft}

            \begin{flushright}
            Texto alinhado à direita
            \end{flushright}

            \begin{center}
            Texto centralizado
            \end{center}
        \end{block}
    \end{columns}
\end{frame}

%% Tipos de arquivo %%
%\subsection{Tipos de arquivo}

%\begin{frame}{Tipos de arquivo}
%\end{frame}

%% Primeiro documento %%
%\subsection{Primeiro documento}

%\begin{frame}{Primeiro documento}
%\end{frame}

%% Inserção de Figuras e Tabelas %%
\subsection{Inserção de Figuras e Tabelas}

% Figuras %
\begin{frame}[fragile]{Figuras}
    \begin{itemize}
    \item Para inserir figuras é necessário utilizar o pacote \textit{graphicx}, ele permite a inserção de gráficos nos mais variados formatos (JPG, BMP, GIF, PS, ...) através do comando:
    \end{itemize}
    
    \alert{\textbackslash includegraphics[\textit{opções}]\{\textit{caminho}\}}

    \begin{itemize}
    \item É necessário inserí-lo dentro do ambiente \textit{figure}.
    \end{itemize}

    \begin{example}
        \begin{semiverbatim}
\small\\begin\{figure\}[h]
\\includegraphics[width=5cm]\{imagem.jpg\}
\\caption\{Minha figura\}
\\label\{fig:01\}
\\end\{figure\}
\end{semiverbatim}
    \end{example}
\end{frame}


% Tabelas %
\begin{frame}[allowframebreaks]{Sintaxe}
    \begin{itemize}
    \item As tabelas são criadas dentro do ambiente \textit{tabular}
    \end{itemize}

    \alert{\textbackslash begin\{\textit{tabular}\}[\textit{posição}]\{\textit{tabulação}\} \\
           \textbackslash end\{\textit{tabular}\}
          }

    \begin{block}{Comandos úteis}
        \begin{description}
        \item[\&] separador de colunas
        \item[\textbackslash\textbackslash] começa nova linha
        \item[\textbackslash hline] linha horizontal
        \item[\textbackslash newline] começa uma nova linha na célula
        \item[\textbackslash cline(\textit{i-j})] linha horizontal da coluna \textit{i} até a \textit{j}
        \item[\textbackslash multicolumn\{\textit{tamanho}\}\{\textit{tabulaçao}\}{...}] define uma célula com múltiplas colunas
        \end{description}
    \end{block}

    \newpage
    
    \begin{description}
    \item[posição:] vertical em referência ao texto em volta, pode ser
        \begin{description}
        \item[b] em baixo
        \item[c] centralizado (default)
        \item[t] em cima
        \end{description}
    \item[tabulação:] determina o alinhamento de cada coluna e as linhas verticais
        \begin{description}
        \item[l] alinhamento esquerdo
        \item[c] centralizado
        \item[r] alinhamento direito
        \item[p\{\textit{largura}\}] coluna com largura definida e com saltos de linha
        \item[|] linha vertical
        \item[||] linha vertical dupla
        \end{description}
    \end{description}
\end{frame}

\begin{frame}[fragile]{Exemplo}
    
    \begin{columns}
    \column{7cm}
    \begin{block}{Entrada}
        \begin{semiverbatim}
\footnotesize\\begin\{tabular\}\{| l | c | r |\}
\\hline
1 \& 2 \& 3 \\\\
\\hline
4 \& 5 \& 6 \\\\
\\hline
7 \& 8 \& 9 
\\hline
\\end\{tabular\}
\end{semiverbatim}
    \end{block}

    \begin{block}{Saída}
        \begin{tabular}{|l|c|r|}
        \hline
        esquerda & centro & direita \\
        \hline
        1 & 2 & 3 \\
        \hline
        4 & 5 & 6 \\
        \hline
        7 & 8 & 9 \\
        \hline
        \end{tabular}
    \end{block}
    \end{columns}
\end{frame}

\begin{frame}[allowframebreaks,fragile]{Ambientes flutuantes}
    O \LaTeX\ possui ambientes ``flutuantes'', isto é, ambientes que são dispostos automaticamente no texto de acordo com seu conteúdo.
    
    Os ambientes Figure e Table fornecem comandos para dinamização dos conteúdos com o documento, como:
    
    \begin{itemize}
    \item título/legenda;
    \item numeração;
    \item referência;
    \item lista de figuras;
    \end{itemize}
    
    \begin{example}
        \begin{semiverbatim}
\\begin\{figure\}[\textit{pos}]
  \\includegraphics\{...\}
  \\caption\{Legenda da figura\}
  \\label\{fig:exemplo\}
\\end\{figure\}

\\begin\{table\}[\textit{pos}]
  \\caption\{Titulo da tabela\}
  \\begin\{tabular\}\{...\}
  ...
  \\end\{tabular\}
  \\label\{tab:exemplo\}
\\end\{table\}
\end{semiverbatim}
    \end{example}
\end{frame}

%%%%%%%%%%%%
%% Aula 3 %%
%%%%%%%%%%%%

\part{Aula 3}
\section{Aula 3}

\subsection{Utilização de Modelos}
\begin{frame}{Utilização de Modelos}
\end{frame}

\subsection{Referências e Bibliografias}
\begin{frame}{Referências e Bibliografias}

  \begin{itemize}
    \item Em \LaTeX\ há diversos métodos para se construir a Bibliografia de um texto, os dois principais são:
 
    \begin{enumerate}
        \item Sistema embarcado
        \item BibTex
    \end{enumerate}
  \end{itemize}

\end{frame}

\begin{frame}[fragile,allowframebreaks]{Sistema Embarcado de Bibliografia}

  \begin{itemize}
    \item Em projetos pequenos, onde a bibliografia não será reutilizada, o método mais eficiente de fazer a bibliografia
    é através do sistema incorporado ao próprio \LaTeX\ através do ambiente \textit{\textbackslash begin\{thebibliography\}}.    
  \end{itemize}

  \begin{block}{}
  \begin{semiverbatim}
    \\begin\{thebibliography\}
    \\bibitem\{lamport94\}

      Leslie Lamport,
      \\emph\{\\LaTeX: A Document Preparation System\}.
      Addison Wesley, Massachusetts,
      2nd Edition,
      1994.

    \\end\{thebibliography\}
  \end{semiverbatim}
  \end{block}

  \begin{itemize}
    \item O comando \textit{thebibliography} deve estar localizado logo acima do \textit{\textbackslash end\{document\}}.
    \item O comando \textit{bibitem} define um item da bibliografia nomeado com o identificador entre os colchetes.
    \item Todo o texto após o \textit{bibitem} será transcrito no arquivo final sem qualquer modificação.
  \end{itemize}

  \begin{itemize}
   \item Para citar um item contido na bibliografia o comando \textit{cite\{cite\_key\}}, onde \textit{cite\_key} é o 
   identificador definido no \textit{bibitem}, deve ser inserido no trecho do texto onde a citação aparecerá.
   \item Para especificar uma página, figura ou teorema da referência, o comando deve ser \textit{cite[especificao]\{cite\_key\}}.
   \item Para múltiplas citações, deve-se user vírgula entre os itens, \textit{cite\{cite\_key1,cite\_key2,cite\_key3\}}.    
  \end{itemize}
 
\end{frame}

\begin{frame}[fragile,allowframebreaks]{BibTex}

  \begin{itemize}
    \item O BibTex funciona como uma pequena base de dados, onde são armazenadas as referências de acordo com uma sintaxe própria e no momento da criação do arquivo final o formato é definido de acordo com o padrão desejado. 
    \item Ao contrário do sistema embarcado, o BibTex utiliza um arquivo diferente do \textit{.tex} original onde está o código \LaTeX.
    \item Uma entrada padrão do BibTex é a seguinte:
  \end{itemize}

  \begin{block}{}
    \begin{semiverbatim}
     @book\{ibrahim,
      address={Rio de Janeiro},
      author={Ibrahim Cesar},
      title={EQM},
      publisher={Osvira Lata},
      year={2008}
    \}
    \end{semiverbatim}
  \end{block}

  \begin{itemize}
   \item Toda entrada BibTex começa com um tipo (lista a seguir). Os tipos são utilizados na padronização da referência no arquivo final. Cada tipo possui determinados campos obrigatórios e opcionais (lista a seguir). 
    \item Para cada tipo de entrada determinados campos são utilizados;
    \item A primeira palavra em um item BibTex depois do tipo é a identificação daquela entrada e deve ser usada toda vez que a referência for usada;
    \item No arquivo contendo o código \LaTeX\ os itens do BibTex devem ser citados utilizando os comandos \textit{\\cite\{citekey\}} ou \textit{\\citeonline\{citekey\}}, onde \textit{citekey} é o identificador do item;
    \item Na prática, os passos para criação de uma bibliografia pelo BibTex são os seguintes:
    \begin{enumerate}
      \item Criar o arquivo .bib;
      \item Gerar o arquivo .aux (rodando \textit{pdflatex} por exemplo);
      \item Rodar o comando \textit{bibtex arquivo} cada vez que o arquivo .bib for modificado;
      \item Rodar novamente o \textit{pdflatex} para criar o arquivo com as referências.
    \end{enumerate}

  \end{itemize}

  \begin{itemize}
    \item Os tipos definidos são:
    \begin{itemize}
      \item \textbf{@article} Um artigo de jornal ou revista;
      \item \textbf{@book} Um livro com uma editora específica;
      \item \textbf{@booklet} Uma obra sem editora ou instituição patrocinadora;
      \item \textbf{@conference} Conferência;
      \item \textbf{@inbook} Parte de um livro, geralmente sem título;
      \item \textbf{@incollection} Parte de um livro com título;
      \item \textbf{@inproceedings} Artigo publicado em anais de conferência;
      \item \textbf{@manual} Documentação técnica;
      \item \textbf{@mastersthesis} Tese de mestrado;
      \item \textbf{@misc} Uso genérico;
      \item \textbf{@phdthesis} Tese de doutorado;
      \item \textbf{@proceedings} Deliberações de uma conferência;
      \item \textbf{@techreport} Um relatório públicado por uma escola ou instituição.
      \item \textbf{@unpublished} Um documento com autor e título, mas não publicado oficialmente.
    \end{itemize}
  
  \end{itemize}

  \begin{itemize}
   \item Os campos disponíveis no BibTex são:
    \begin{itemize}
      \item \textit{address}: endereço do editor, geralmente a cidade;
      % \item \textit{annote}: anotação para estilos de bibliografia;
      \item \textit{author}: autor, em caso de mais de um, separado por \textit{and};
      \item \textit{booktitle}: Título do livro;
      \item \textit{chapter}: Capítulo;
      \item \textit{crossref}: Chave de entrada para referência cruzada;
      \item \textit{edition}: Edição;
      \item \textit{editor}: Editor;
      \item \textit{eprint}: Especificação de uma publicação eletrônica;
      \item \textit{howpublished}: Como foi publicado, caso não usual;
      \item \textit{institution}: Instituição envolvida na edição;
      \item \textit{journal}: Jornal ou Revista da publicação;
      \item \textit{key}: Campo oculto, usado na classificação alfabética das referências quando \textit{author} e \textit{editor} estão ocultos;
      \item \textit{month}: Mês de publicação;
      \item \textit{note}: Informação extra;
      \item \textit{number}: Número (edição) de um Jornal ou Revista;
      \item \textit{organization}: Patrocinador de uma conferência;
      \item \textit{pages}: Páginas;
      \item \textit{publisher}: Editora;
      \item \textit{school}: Instituição de Ensino onde a tese foi escrita; 
      \item \textit{series}: Série de um livro;
      \item \textit{title}: Título do trabalho;
      \item \textit{type}: Tipo de relatório;
      \item \textit{url}: Endereço WWW.
      \item \textit{volume}: Volume para uma obra multi-volume;
      \item \textit{year}: Ano de publicação.

    \end{itemize}

  \end{itemize}


  



 
\end{frame}


\subsection{Divisão de aquivos}
\begin{frame}{Divisão de arquivos}

    \begin{itemize}
    \item Em projetos com grande quantidade de texto pode ser interessante separar o documento em diferentes arquivos para melhor organização através do comando:    
    \end{itemize}

    \alert{\textbackslash include\{\textit{nomedoarquivo.tex}\}}

    \begin{itemize}
     \item Dentro do documento \LaTeX, no ambiente \textit{\textbackslash begin\{document\}} no momento em que aparece o comando o processamento vai imediatamente buscar o texto contido no arquivo.
    \end{itemize}

    \begin{itemize}
     \item Caso o arquivo esteja em uma pasta diferente da do arquivo principal, deve ser especificado o caminho completo do arquivo que será incluído.
    \end{itemize}

\end{frame}




%%%%%%%%%%%%
%% Aula 4 %%
%%%%%%%%%%%%

\part{Aula 4}
\section{Aula 4}

\subsection{Apresentações (BEAMER)}

\begin{frame}[fragile]{Introdução}
    \begin{itemize}
    \item {\footnotesize BEAMER} é uma classe do \LaTeX\ para criação de apresentações de slides ou transparências.
    \item Uma apresentação em {\footnotesize BEAMER} é criada como qualquer outro documento \LaTeX, diferentes slides são colocados em ambientes (chamados \textit{frames}).
    \item Muitos pacotes de \LaTeX\ já contêm a classe {\footnotesize BEAMER}, no entanto, sua versão atualizada pode ser encontrada em: \url{http://bitbucket.org/rivanvx/beamer}
    \end{itemize}
\end{frame}

\begin{frame}[fragile]{Vantagens}
    \begin{itemize}
    \item Pode ser usado com \verb|pdflatex|, \verb|latex+dvips|, \verb|luatex| e \verb|xelatex|.
    \item Efeitos e sobreposições podem ser criados facilmente.
    \item Sua estrutura torna fácil criar apresentações de outras classes como \verb|article| e \verb|book|.
    \item A saída final é um arquivo PDF, assim não é preciso se preocupar se determinado programa está instalado em diferentes locais.
    \end{itemize}
\end{frame}

\begin{frame}[fragile]{Estrutura básica}
    Uma apresentação em {\footnotesize BEAMER} tem a mesma estrutura de um documento \LaTeX.
    
{\footnotesize
\begin{verbatim}
    \documentclass{beamer}

    \usetheme{Darmstadt}

    \title{Titulo da apresentação}
    \author{Nome do autor}
    \institut{Instituição}
    \date{Data da apresentação}

    \begin{document}
        \frame{\pagetitle}

        \begin{frame}
            Minha apresentação
        \end{frame}
    \end{document}
\end{verbatim}
}
\end{frame}

\begin{frame}[fragile]{Frames}
    \begin{itemize}
    \item Frame é o ambiente onde se cria um ou uma sequência de slides.
    \item[] \alert{\textbackslash begin\{\textit{frame}\}[\textit{opções}]\{\textit{Título}\}}
    \item Eles também podem ser criados com o comando \alert{\textbackslash frame\{\}}.
    \item Se o conteúdo exceder um slide, a opção \alert{allowframebreaks} pode ser utilizada para dividir o frame em várias partes.
    \item Se a divisão não ficar como desejado, também pode-se usar o comando \alert{\textbackslash newpage} para começar um novo slide a partir deste ponto.
    \end{itemize}
\end{frame}

\begin{frame}{Página título}
\end{frame}

\begin{frame}{Itemização, numeração e descrição}
\end{frame}

\begin{frame}{Sobreposições e efeitos}
\end{frame}

\begin{frame}{Destacando textos}
\end{frame}

\begin{frame}{Ambiente Block}
\end{frame}

\begin{frame}{Figuras e tabelas}
\end{frame}

\begin{frame}{Colunas}
\end{frame}


\subsection{Fórmulas Matemáticas}
\begin{frame}{Fórmulas Matemáticas}

\end{frame}

\end{document}
