%%%%%%%%%%%%
%% Aula 2 %%
%%%%%%%%%%%%

\part{Aula 2}
\section{Aula 2}

%% Arquivo de Entrada %%
\subsection{Arquivo de entrada}

\begin{frame}{Arquivo de entrada}
	\begin{itemize}
	\item A entrada para o \LaTeX\ é um arquivo de texto, ASCII ou UTF-8. Sendo possível criá-lo em qualquer editor de texto.
	\end{itemize}
	
	\alert{\it nomedoarquivo.tex}

    \pause

    {\it nomedoarquivo.bib (opcional) }

	\vspace{.5cm}
	\pause

	\begin{itemize}
	\item Arquivos auxiliares (gerados automaticamente):
	\end{itemize}
	
	{\it nomedoarquivo.\{aux, log, nav, out, snm, toc, \dots\} }
\end{frame}

%% Estrutura básica %%
\subsection{Estrutura Básica}

% Comandos %
\begin{frame}{Comandos}
	\begin{itemize}
	\item Os comandos no \LaTeX\ têm a seguinte forma:
	\end{itemize}

	\alert{\textbackslash nomedocomando}

	\begin{exampleblock}{Exemplo}
		\textbackslash LaTeX \\ \textbackslash alpha
	\end{exampleblock}
	
	\vspace{.5cm}

	\alert{\textbackslash nomedocomando\{\textit{parâmetros}\} }

	\begin{exampleblock}{Exemplo}
		\textbackslash section\{Introdução\} \\ \textbackslash sqrt\{x\}
	\end{exampleblock}
\end{frame}

% Caracteres Especiais %
\begin{frame}{Caracteres especiais}
	\begin{itemize}
	\item Os seguintes símbolos são caracteres especiais no \LaTeX:
	\end{itemize}

	\alert{\# \$ \% \^{} \& \_ \{ \} \~{} \textbackslash}

	\begin{itemize}
	\item Para utilizá-los é necessário a inserção de ``\textbackslash'' antes:
	\end{itemize}

	\alert{
	   \textbackslash\# 
	   \textbackslash\$ 
	   \textbackslash\% 
	   \textbackslash\^{} 
	   \textbackslash\& 
	   \textbackslash\_ 
	   \textbackslash\{ 
	   \textbackslash\} 
	   \textbackslash\~{} 
	   \textbackslash textbackslash
	}
\end{frame}

%  Espaços e parágrafos %
\begin{frame}[fragile,allowframebreaks]{Espaços e parágrafos}
	\begin{itemize}
	\item Apenas um espaço em branco é considerado pelo \LaTeX.
	\item O comando \alert{\textbackslash\textbackslash} faz uma quebra-de-linha.
	\item Uma linha em branco representa um novo parágrafo.
	\end{itemize}

    \begin{block}{Entrada}
        \small
        \begin{semiverbatim}
\small{}Não faz diferença um ou mais        espaços depois de uma
palavra. Com \\\\ o texto vai para a próxima linha.

E uma linha em branco representa um novo parágrafo.
\end{semiverbatim}
    \end{block}    
    
    \begin{block}{Saída}
        \small
        Não faz diferença um ou mais       espaços depois de uma palavra. Com \\ o texto vai para a próxima linha.

        E uma linha em branco representa um novo parágrafo.
    \end{block}
    
    \begin{itemize}
    \item Para adicionar mais espaços pode-se usar os comandos:
        \begin{description}
        \item[\textbackslash ,] um caracter de espaço (\,)
        \item[\textbackslash \textvisiblespace] um espaço mediano (\ )
        \item[\textbackslash quad] um tab (\quad)
        \item[\textbackslash qquad] dois tabs (\qquad)
        \end{description}
    \end{itemize}
\end{frame}

% Espaçamentos %
\begin{frame}{Espaçamento Vertical}
    \begin{itemize}
    \item O \LaTeX determina automaticamente os espaços entre dois parágrafos, itens, figuras, \dots Em casos especiais, que necessite um espaçamento maior, pode-se usar o comando
    \item[] \alert{\textbackslash vspace\{\textit{comprimento}\}}
    
    \vspace{.5cm}

    \item Este comando deve ser utilizado sempre entre duas linhas vazias.
    \item O comprimento pode ser dado em cm, mm, in, pt, e outras.
    \end{itemize}
\end{frame}

\begin{frame}[fragile]{Espaçamento Horizontal}
    \begin{itemize}
    \item Do mesmo modo que se pode definir o espaçamento vertical, pode-se definir o espaçamento horizontal
    \item[] \alert{\textbackslash hspace\{\textit{comprimento}\}}

    \vspace{.5cm}

    \item Diferentemente do \verb|vspace|, o \verb|hspace| pode ser utilizado entre o texto.
    \end{itemize}
\end{frame}

%% Estrutura do arquivo %%
\subsection{Estrutura do arquivo}

% Classe %
\begin{frame}[allowframebreaks]{Classe}
	\begin{itemize}
	\item Primeiro comando do arquivo deve ser o tipo do documento, ou classe, que é feito pelo seguinte comando:
	\end{itemize}

	\alert{\textbackslash documentclass[\textit{opções}]\{\textit{classe}\}}

	\vspace{.5cm}
	
	\begin{block}{Mais usados}
		\begin{center}
		\begin{tabular}{|l|l|}
		\hline
		{\bf Opções}				& {\bf Classes} \\
		\hline
		10pt, 11pt, 12pt			& article		\\
		a4paper, letterpaper, ...	& report		\\
		onecolumn, twocolumn		& book			\\
		twoside, oneside			& beamer		\\
		\vdots						& \vdots		\\
		\hline
		\end{tabular}
		\end{center}
	\end{block}
\end{frame}

% Pacotes %
\begin{frame}{Pacotes}
	\begin{itemize}
	\item É possível adicionar pacotes para aumentar as funcionalidades do \LaTeX\, como cores, figuras, fontes, etc. Para isso	se usa o comando:
	\end{itemize}

	\alert{\textbackslash usepackage[\textit{opções}]\{\textit{pacote}\}}

	\vspace{.5cm}

	\begin{exampleblock}{Exemplo}
		\begin{itemize}
		\item \textbackslash usepackage[\textit{brazil}]\{\textit{babel}\}
		\item \textbackslash usepackage[\textit{utf8}]\{\textit{inputenc}\}
		\item \textbackslash usepackage[\textit{T1}]\{\textit{fontenc}\}
		\item \textbackslash usepackage\{\textit{graphicx}\}
		\end{itemize}
	\end{exampleblock}
\end{frame}

% Corpo do texto %
\begin{frame}[fragile]{Corpo do texto}
    \begin{itemize}
    \item Após feitas as configurações, o corpo do texto é iniciado pelo comando
    \end{itemize}

    \alert{\textbackslash begin\{\it document\}}
    
    \begin{itemize}
    \item E finalizado por
    \end{itemize}

    \alert{\textbackslash end\{\it document\}}

    \begin{exampleblock}{Exemplo}
        \begin{semiverbatim}
\\documentclass[11pt,a4paper]\{article\}
\\usepackage[brazil]\{babel\}
\\usepackage\{amsmath\}
...
\\begin\{document\}
...
\\end\{document\}
\end{semiverbatim}
    \end{exampleblock}
\end{frame}

%% Comandos Globais de Configuração %%
\subsection{Comandos Globais de configuração}

% Capítulos, seções, subseções %
\begin{frame}{Capítulos, seções, subseções}
    \begin{itemize}
    \item O \LaTeX\ suporta até três níveis de seção e dois níveis de parágrafo na classe ``article'':
    \end{itemize}
   
    \begin{columns}
    \column{.5\textwidth}
    \alert{\quad \textbackslash section\{...\} \\
           \quad \textbackslash subsection\{...\} \\
           \quad \textbackslash subsubsection\{...\} \\
           \quad \textbackslash paragraph\{...\} \\
           \quad \textbackslash subparagraph\{...\} \\
        }
    \column{.5\textwidth}
    \small
    obs: Caso não queira que apareça a numeração é só colocar um * antes das chaves. Ex: \textbackslash section*\{\dots\}
    \end{columns}

    \begin{itemize}
    \item Em adicional, nas classes ``report'' e ``book'' há um seção superior
    \end{itemize}

    \alert{\textbackslash chapter\{...\}}
\end{frame}

% Título e Sumário %
\begin{frame}{Título e sumário}
    \begin{itemize}
    \item Para gerar o título do documento usa-se
    \end{itemize}

   \alert{\textbackslash maketitle} (opcional)

    \begin{itemize}
    \item E seu conteúdo é definido pelo seguintes comandos
    \end{itemize}

    \alert{\textbackslash title\{...\}} \\
    \alert{\textbackslash author\{...\}} \\
    \alert{\textbackslash date\{...\}} {\small (opcional)}

    \begin{itemize}
    \item Para construir o sumário, lista de figuras ou tabelas é necessário apenas um comando
    \end{itemize}
    
    \alert{\textbackslash tableofcontents \\
           \textbackslash listoffigures \\
           \textbackslash listoftables
        }

\end{frame}

% Ambientes %
\begin{frame}{Ambientes}
    \begin{itemize}
    \item Para compor textos com algúm propósito especial o \LaTeX\ define muitos tipos de ambientes
    \end{itemize}

    \alert{\textbackslash begin\{\it ambiente\} \\
           \textit{texto} \\
           \textbackslash end\{\it ambiente\}}

    \begin{itemize}
    \item É possível colocar vários ambientes um dentro do outro
    \end{itemize}

    \alert{\small
    \textbackslash begin\{aaa\} \\
    \ \ \textbackslash begin\{bbb\} \\
    \ \ \ \ \textbackslash begin\{ccc\} \\
    \ \ \ \ \vdots \\
    \ \ \ \ \textbackslash end\{ccc\} \\
    \ \ \textbackslash end\{bbb\} \\
    \textbackslash end\{aaa\}
    }
\end{frame}

% Listagens %
\begin{frame}[fragile]{Listagens}
    \begin{itemize}
    \item Existem três ambientes básicos para listagens:
        \begin{description}
        \item[itemize:] listas simples
        \item[enumerate:] listas enumeradas
        \item[description:] descrições
        \end{description}
    \end{itemize}
    
    \begin{columns}
        \column{.5\textwidth}
        \begin{block}{Entrada}
            \begin{semiverbatim}
\footnotesize{}\\begin\{enumerate\}
\\item Primeiro
  \\begin\{itemize\}
  \\item Com ponto
  \\item[-] Com traço
  \\end\{itemize\}
\\item Segundo
  \\begin\{description\}
  \\item[Item] descrição
  \\end\{description\}
\\end\{enumerate\}
\end{semiverbatim}
        \end{block}
    
        \column{.5\textwidth}
        \begin{block}{Saída}
            \begin{enumerate}
            \item[1] Primeiro
            \begin{itemize}
            \item Com ponto
            \item[-] Com traço
            \end{itemize}
            \item[2] Segundo
            \begin{description}
            \item[Item] descrição
            \end{description}            
            \end{enumerate}
        \end{block}
    \end{columns}
\end{frame}

% Alinhamento %
\begin{frame}[fragile]{Alinhamento}
    \begin{itemize}
    \item Existem três ambientes básicos para alinhamento:
        \begin{description}
        \item[flushleft:] alinha o texto a esquerda (default)
        \item[flushright:] alinha o texto a direita
        \item[center:] centraliza o texto
        \end{description}
    \end{itemize}
    
    \begin{columns}
        \column{.5\textwidth}
        \begin{block}{Entrada}
            \begin{semiverbatim}
\footnotesize{}\\begin\{flushleft\}
Texto alinhado à esquerda.
\\end\{flushleft\}

\\begin\{flushright\}
Texto alinhado à direita.
\\end\{flushright\}

\\begin\{center\}
Texto centralizado.
\\end\{center\}
\end{semiverbatim}
        \end{block}
    
        \column{.5\textwidth}
        \begin{block}{Saída}
            \begin{flushleft}
            Texto alinhado à esquerda
            \end{flushleft}

            \begin{flushright}
            Texto alinhado à direita
            \end{flushright}

            \begin{center}
            Texto centralizado
            \end{center}
        \end{block}
    \end{columns}
\end{frame}

% Minipage %
\begin{frame}[fragile]{Minipage}
    \begin{itemize}
    \item O ambiente \textit{minipage} simula uma minipágina, ou janela, em uma posição da página. É útil para dividir trechos do texto em colunas, fazer comparações ou inserir figuras e tabelas lado a lado.
    \item[] \alert{\textbackslash begin\{\textit{minipage}\}[\textit{alinhamento}]\{\textit{largura}\}}
    \item A \textit{largura} pode ser dada em relação à largura do texto \verb|\textwidth| ou em unidade fixa (cm, pt) e o \textit{alinhamento} pode ser \textit{c}, \textit{b} ou \textit{t}, referente ao texto ao seu redor.
    \end{itemize}

    \begin{exampleblock}{Exemplo}
\begin{semiverbatim}\footnotesize
    \\begin\{minipage\}[t]\{.5\\textwidth\}
        Coluna 1.
    \\end\{minipage\}
    \\begin\{minipage\}[t]\{.5\\textwidth\}
        Coluna 2.
    \\end\{minipage\}
\end{semiverbatim}
    \end{exampleblock}
\end{frame}

%% Inserção de Figuras e Tabelas %%
\subsection{Inserção de Figuras e Tabelas}

% Figuras %
\begin{frame}[allowframebreaks,fragile]{Figuras}
    \begin{itemize}
    \item Para inserir figuras é necessário utilizar o pacote \textit{graphicx}, ele permite a inserção de gráficos nos mais variados formatos (JPG, BMP, GIF, PS, ...) através do comando:
    \end{itemize}
    
    \alert{\textbackslash includegraphics[\textit{opções}]\{\textit{caminho}\}}

    \begin{itemize}
    \item É necessário inserí-lo dentro do ambiente \textit{figure}.
    \end{itemize}

    \begin{exampleblock}{Exemplo}
        \begin{semiverbatim}
\small\\begin\{figure\}[h]
\\includegraphics[width=5cm]\{imagem.jpg\}
\\caption\{Minha figura\}
\\label\{fig:01\}
\\end\{figure\}
\end{semiverbatim}
    \end{exampleblock}

    \begin{itemize}
    \item Com o pacote \verb|subfigure| podemos inserir mais de uma figura simultaneamente com o comando \alert{\textbackslash subfigure}
    \end{itemize}

    \begin{exampleblock}{Exemplo}
    \begin{semiverbatim}
\footnotesize\\usepackage\{subfigure\}

\\begin\{figure\}[!htb]
  \\begin\{center\}
    \\subfigure[desc (a) ]\{\\includegraphics\{fig01.jpg\}\}\\quad
    \\subfigure[desc (b)]\{\\includegraphics\{fig02.jpg\}\} \\\\
    \\subfigure[desc (c)]\{\\includegraphics\{fig03.jpg\}\} \\quad
    \\subfigure[desc (d)]\{\\includegraphics\{fig04.jpg\}\}
  \\caption\{Caption da figura.\}\\label\{fig:exemploDeSubfigure\}
  \\end\{center\}
\\end\{figure\}
\end{semiverbatim}
    \end{exampleblock}
\end{frame}


% Tabelas %
\begin{frame}[allowframebreaks]{Tabelas}
    \begin{itemize}
    \item As tabelas são criadas dentro do ambiente \textit{tabular}
    \end{itemize}

    \alert{\textbackslash begin\{\textit{tabular}\}[\textit{posição}]\{\textit{tabulação}\} \\
           \textbackslash end\{\textit{tabular}\}
          }

    \begin{block}{Comandos úteis}
        \begin{description}
        \item[\&] separador de colunas
        \item[\textbackslash\textbackslash] começa nova linha
        \item[\textbackslash hline] linha horizontal
        \item[\textbackslash newline] começa uma nova linha na célula
        \item[\textbackslash cline\{\textit{i-j}\}] linha horizontal da coluna \textit{i} até a \textit{j}
        \item[\textbackslash multicolumn\{\textit{tamanho}\}\{\textit{tabulaçao}\}{...}] define uma célula com múltiplas colunas
        \end{description}
    \end{block}

    \newpage
    
    \begin{description}
    \item[posição:] vertical em referência ao texto em volta, pode ser
        \begin{description}
        \item[b] em baixo
        \item[c] centralizado (default)
        \item[t] em cima
        \end{description}
    \item[tabulação:] determina o alinhamento de cada coluna e as linhas verticais
        \begin{description}
        \item[l] alinhamento esquerdo
        \item[c] centralizado
        \item[r] alinhamento direito
        \item[p\{\textit{largura}\}] coluna com largura definida e com saltos de linha
        \item[|] linha vertical
        \item[||] linha vertical dupla
        \end{description}
    \end{description}
\end{frame}

\begin{frame}[fragile]{Exemplo}
    
    \begin{columns}
    \column{7cm}
    \begin{block}{Entrada}
        \begin{semiverbatim}
\footnotesize\\begin\{tabular\}\{ l | c | r \}
esquerda \& centro \& direita \\\\
\\hline
1 \& 2 \& 3 \\\\
\\cline\{2-2\}
4 \& 5 \& 6 \\\\
\\cline\{2-2\}
7 \& 8 \& 9 \\\\
\\hline
\\end\{tabular\}
\end{semiverbatim}
    \end{block}

    \begin{block}{Saída}
    \footnotesize{
        \begin{tabular}{l|c|r}
        esquerda & centro & direita \\
        \hline
        1 & 2 & 3 \\
        \cline{2-2}
        4 & 5 & 6 \\
        \cline{2-2}
        7 & 8 & 9 \\
        \hline
        \end{tabular}
    }
    \end{block}
    \end{columns}
\end{frame}

\begin{frame}[allowframebreaks,fragile]{Ambientes flutuantes}
    \begin{itemize}
    \item O \LaTeX\ possui ambientes ``flutuantes'', isto é, ambientes que dispõem o texto automaticamente de acordo com seu conteúdo.
    \item Os ambientes Figure e Table fornecem comandos para dinamização dos conteúdos com o documento, como:
        \begin{itemize}
        \item título/legenda;
        \item numeração;
        \item referência;
        \item lista de figuras;
        \end{itemize}
    \end{itemize}

    \begin{exampleblock}{Exemplo}
        \begin{semiverbatim}
\\begin\{figure\}[\textit{posição}]
  \\includegraphics\{...\}
  \\caption\{Legenda da figura\}
  \\label\{fig:exemplo\}
\\end\{figure\}

\\begin\{table\}[\textit{posição}]
  \\caption\{Titulo da tabela\}
  \\begin\{tabular\}\{...\}
  ...
  \\end\{tabular\}
  \\label\{tab:exemplo\}
\\end\{table\}
\end{semiverbatim}
    \end{exampleblock}

    \begin{itemize}
    \item Caso o texto tenha 2 colunas, mas você quer a figura ou a tabela usando a página inteira, coloca-se um asterisco:
    \item[] {\small \textbackslash begin\alert{*}\{figure\}}
    \item[] {\small \textbackslash end\{figure\}}

    \vspace{.5cm}

    \item A \textit{posição} pode ser \verb|h|, \verb|t| e \verb|b|, para ``forçar'' a posição utiliza-se uma exclamação.
    \item[] {\small \textbackslash begin\{figure\}[\alert{!}h]}
    \end{itemize}
\end{frame}


