%%%%%%%%%%%%
%% Aula 1 %%
%%%%%%%%%%%%

\part{Aula 1}
\section{Aula 1}

%% O que é LaTeX %%
\subsection{O que é \LaTeX?}

\begin{frame}{O que é \LaTeX?}
    \begin{itemize}
    \item \LaTeX (ou "LaTeX") é um sistema de preparação de documentos com alta qualidade tipográfica.
    \item É comumente utilizado para médios a largos documentos técnicos ou científicos, mas pode ser utilizado para quase qualqer tipo de publicação.
    \item \LaTeX é pronunciado "Lay-tech" ou "Lah-tech"
    \end{itemize}
\end{frame}

%% Histórico %%
\subsection{Histórico}

\begin{frame}{Histórico}
    \begin{itemize}
    \item Em 1978 Donald E. Knuth começou a desenvolver uma linguagem cujo objetivo era permitir a qualquer um formatar textos com muitas equações e com alta qualidade de saída, chamada de \TeX.
    \item Em 1985 Leisle Lamport desenvolveu um conjunto de macros denominado \LaTeX, que simplifica o uso da linguagem \TeX.
    \item Agora este projeto é mantido e desenvolvido pelo \LaTeX3 Project.
    \end{itemize}
\end{frame}

%% Conceituação %%
\subsection{Conceituação}

\begin{frame}{Conceituação}
    \begin{itemize}
    \item<alert@1> O \LaTeX\ não é um processador de textos!
    \item O \LaTeX\ encoraja o autor a não se preocupar muito com a aparência e se focar na preparação do mesmo.
    \item Entretanto, algumas ferramentas — como o LyX — combinam o \LaTeX\ com a sistemática do WYSIWYG (What You See Is What You Get).
    \end{itemize}
\end{frame}

%% Vantagens e Desvantagens %%
\subsection{Vantagens e Desvantagens}

\begin{frame}[allowframebreaks]{Vantagens e Desvantagens}
    \begin{columns}
    \column{12cm}
    \begin{block}{Vantagens}
        \begin{itemize}
        \item Aparência profissional;
        \item A atenção dos usuários se concentra no conteúdo e não na aparência;
        \item Possibilidade de fácil utilização, devido ao uso de ferramentas gráficas como Kyle e LyX;
        \item A edição de fórmulas matemáticas é robusta e sua apresentação, visualmente agradável;
        \item Facilidade na criação de estruturas complexas como bibliografia, notas de rodapé, sumário e citações estão abstraídas;
        \item Ambos \TeX\ e \LaTeX\ são programas livres;
        \item Utilização modesta dos recursos do sistema.
        \end{itemize}
    \end{block}
    \end{columns}

    \begin{block}{Desvantagens}
        \begin{itemize}
        %\item \LaTeX\ does not work well for people who have sold their souls ...(haha)
        \item A principal desvantagem é que a criação de novos modelos leva muito tempo;
        \item A aprendizagem é mais difícil que em programas WYSIWYG, pois embora a estrutura lógica do documento seja intuitiva, os comandos do LaTeX, obviamente, não o são.
        \end{itemize}
    \end{block}
\end{frame}

%% Instalação no Linux %%
\subsection{Instalação no Linux}

\begin{frame}{Instalação no Linux}
    \begin{itemize}
    \item Depende de cada distribuição, sendo o caminho mais fácil procurar pelos pacotes \textit{texlive} e \textit{latex-beamer} no gerenciador de pacotes da distribuição utilizada.
    \end{itemize}
\end{frame}

%% Instalação no Windows %%
\subsection{Instalação no Windows}

\begin{frame}{Instalação no Windows}
    Passos para a instalção do \LaTeX\ no Windows:
    
    \begin{enumerate}
    \item Baixar o MikTex (Ambiente \LaTeX\ para windows)
    \item Baixar o WinEdt (Um editor \LaTeX, o melhor para windows, porém é shareware). Ou o TeXnicCenter (que é gratuito) ou qualquer outro editor de texto.
    \item Instalá-los na mesma ordem.
    \end{enumerate}
\end{frame}


